% ============================
%          ABSTRACT
% ============================

\chapter*{abstract}
This thesis describes the development and applications of a laser turret with pan and tilt control: this device can be used to project a laser dot on a given surface (wall and/or floor) and finely control its position by solving the system’s inverse kinematics.

Once its functionality was validated, we used the turret for an human-robot interaction task.  In particular, we considered an existing system in which an operator interacts with a drone using pointing gestures; the system initially determines the relative localization between the two, then allows the operator to control the drone, which follows the indicated location in real time. The existing approach relied on a fast agile robot, and was unsuitable for implementation on slower or larger ground robots.  In this thesis, we demonstrate how the turret can be used with this goal.

Finally, the turret was adopted to efficiently run experiments for fine tuning or validating different components of the system described above, such as the algorithms for relative localization and the algorithms for reconstruction of the pointed direction.  To this end, we ran an experimental campaign involving ten users.
% ============================
%        INTRODUCTION
% ============================
\chapter*{Introduction\markboth{Introduction}{}} 
\addcontentsline{toc}{chapter}{Introduction}
TODO

