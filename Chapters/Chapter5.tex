\chapter{Conclusions and Future Work} 
\label{chap:5} 
\section{Conclusions}
\subsection{The Turret}
The first step of this project is to develop a pan and tilt turret able to project a laser dot on a given surface and finely control its position by solving the inverse kinematics. That goal is achieved with two different models, that we have described in that thesis both in terms of geometrical model and hardware/software implementations. The result is that, with our first turret, we are able to fulfill all our goals in a proper way. However, the second model proves to be better in performances, since it allows us to control the laser with greater precision and smoothness, being built with more expensive and powerful servos.

\subsection{HRI Task}
With the functionality of the turret validated, we are able to deploy our system for a \ac{HRI} task. In fact, we integrate that work with an existing system in which an operator interacts with a drone using pointing gestures \cite{gromov2018robot}. That interaction approach is suitable for fast agile robot, able to follow indicated locations in real time, but does not fit for slower or larger ground robots. Our goal is to provide a usable interface for those latter cases with our turret. So, by mounting the turret on a kobuki, we are able to solve the robot navigation task efficiently, applying the relative localization approach to the aforementioned slow ground robot and driving it around exploiting the laser pointing. Applications developed show that our approach works well in practice, as users are able to drive the robot to a goal point or through a trajectory very precisely.

\subsection{Experiments}
Finally, the turret gives us the possibility to efficiently run experiments to validate our system components. In particular, we report two experiments: pointing experiment and \ac{relloc} experiment.

The first helps us to understand whether our over-simplified human pointing model is good enough to work well. The answer is yes, as results show. Moreover, it also points out the fact that users, with a feedback for their pointing, are able to compensate any intrinsic error or misalignment of the system, giving a strong confirmation to our work. This is not surprising, because is exactly one of the motivations that we highlights analyzing related works in \textit{\nameref{sec:related}} section.

The second is useful to asses the precision of the relative localization procedure. Moreover, that experiment was run with a group of unskilled people, so it is also an indicator of how easy the procedure is to be followed. Results are quite good, considering the fact that we had time to run the data collection campaign only with the first turret, which is less accurate. Demos we developed later, which deployed the \ac{relloc} with the new turret, give us qualitative evidences that this model is more precise and reliable. We even found out that performing the \ac{relloc} sometimes is more precise than manually setting the position of the user. This happens because measuring manually is of course prone to human error and, most important thing, does not take into account any inaccuracy of the system. The \ac{relloc} thing, on the contrary, takes place entirely within the system environment, so, when it is precise, is able to collocate the human within the system accurately.

\section{Future Work}
\subsection{Relloc and Pointing Experiments}
As already explained many times, the \ac{relloc} experiment was run only with the first turret model. We plan to conduct a data collection campaign with the second model. Being more precise, we expect to obtain better results and a more reliable indicator of the \ac{relloc} accuracy. If that intuition will prove to be true, we could also rethink the pointing experiment in such a way that we do not set the position of the human by hand, but we make him run the \ac{relloc} before starting with the data collection.
\subsection{Human Body Measures}
To allow users to use the system (e.g. to run experiments) we had to collect their physical data in order to build the human system properly. Since that procedure is time consuming and error prone, we would like to make it automatic. For example, we could determine approximate measures based on user's height and sex. Moreover, we could exploit the \ac{IMU} sensor to determine the height. That procedure would make interfacing with the system easier and faster for new users.

\subsection{Point to Wall (in Arbitrary Places)}
As already explained in section \ref{subs:floorwall}, we implemented the pipeline also to drive the laser on the wall, but there are some limitations. In fact, we have to explicitly set the distance of the turret from the wall and its relative orientation. A possible further development could help to remove that requirement. For example, we could equip the turret with a coaxial laser rangefinder with pointer. In that way, we could be able to determine those initial condition autonomously and thus the wall would not need to be \virgolette{calibrated} manually. That would allow the system to be deployed in arbitrary places.